\documentclass[a4paper, 10pt]{article}

\usepackage[margin = 1in]{geometry} % for spacing around
\usepackage{graphicx} % for including images in your pdfs
\usepackage{xcolor} % for including colors in your pdf
\usepackage{soul} % for text decoration
\usepackage[utf8]{inputenc} % for encoded text
\usepackage[T1]{fontenc}
\usepackage{setspace} % for setting different line spacings between paragrafs.
\usepackage{enumerate} % for letting us get more detailed enumerate lists
\usepackage{multirow} % to let us combine more rows together
\usepackage{colortbl} % for decorating tables
\usepackage{amsmath} % used for representing more complicated math displays
\usepackage{supertabular}
\usepackage{longtable} % both of these packages are used to making really big tables
\usepackage{wrapfig} % allows us to wrap text around figures
\usepackage{fancyhdr} % for making fancy headers
%\usepackage{bibtex} % for making better bibliographies
\usepackage[pdftex]{hyperref} % for letting us make links
\usepackage{lscape} % Allows us to flip from portrait to landspace
\usepackage{tikz} % for high detailed drawing
\usepackage{multicol} % To put things side by side
\usepackage{rotating} % For rotating objects
% \usepackage{draftwatermark} % For adding watermarks
\usepackage{MnSymbol} % for using multiple symbols
\usepackage{mathtools} % Used for more math symbols
\usepackage{xfrac} % For more complciated fractions and to add derivitives
\usepackage{enumitem} % for better enum lists
\usepackage{tcolorbox} % for adding colored text boxes
\usepackage{bm} % Adding bold text to math inputs
\usepackage{pgfplots} % Used for plotting functions
\usepackage{wallpaper} % For adding wallpapers

% Setting up the default image path
\graphicspath{{./Images/}}

% Implementing authro details
\title{Introduction to Haskell Programming Language}
\author{Emre Arapcic-Uevak}
\date{}

% Setting up the fancy page style
\fancypagestyle{customStyle}{
	\lhead{} \chead{} \rhead{}
	\lfoot{} \cfoot{\thepage} \rfoot{}
	\renewcommand{\headrulewidth}{0pt}
	\renewcommand{\footrulewidth}{1pt}
}
\pagestyle{customStyle}

% Setting up hyperref options
\hypersetup {
	colorlinks = false,
	citecolor = black,
	filecolor = blue,
	linkcolor = blue,
	urlcolor = blue,
	pdftex
}

% Custom commands


\begin{document}
	\maketitle
	\vspace{5mm}
	
	\begin{abstract}
		\begin{center}
			This presentation introduces the Haskell programming language, covering its history, key features, data structures, comparisons with other languages, and providing simple code examples. The audience will gain insights into the fundamentals of Haskell and its applications in modern programming.
		\end{center}
	\end{abstract}

	\ThisCenterWallPaper{.4}{Images/IUS Logo.png}

	\pagebreak
	\tableofcontents
	\pagebreak

	\section{Introduction to Haskell}
        \noindent Haskell is a purely functional programming language, distinguished by its strong static typing, high level of abstraction, and lazy evaluation. Inspired by the principles of lambda calculus, Haskell emphasizes functions without side effects and immutable data. Named after the logician Haskell Curry, it stands apart from imperative programming languages by treating computation as the evaluation of mathematical functions, thereby avoiding changing-state and mutable data.
	\section{History and Evolution of Haskell}
        \subsection{The Foundational Principles}
            Haskell's journey begins with its roots deeply embedded in lambda calculus, a sophisticated mathematical system devised by Alonzo Church in 1932. This system, focusing on function abstraction and application, forms the core of Haskell's functional programming ethos.

        \subsection{The Dawn of Functional Programming}
            Tracing its lineage to the 1960s, Haskell shares its heritage with LISP, the first functional language, born in 1959. However, Haskell emerged as a response to the limitations of its predecessors like ML, Hope, and Miranda, which, despite their innovations, struggled in practical application and widespread adoption.

        \subsection{The Formation of Haskell}
            Haskell's story officially starts at the FPCA '87 conference in Portland, Oregon. Here, a visionary group of academics united to forge a new path in programming language design. Their collaboration led to the first Haskell report in 1990, laying out the motivations, aspirations, and the unique nature of Haskell.

        \subsection{Milestones in Haskell’s Journey}
            Haskell's evolution saw pivotal moments, such as the release of the "Gentle Introduction to Haskell" tutorial and the advent of its essential tools – the GHC compiler and GHCI interpreter in 1992. At a seminal meeting at Yale in 1988, Haskell's goals were crystallized, focusing on research, teaching, and the development of robust systems.

        \subsection{A Tribute to Haskell B. Curry}
            Named to honor the eminent logician and mathematician Haskell B. Curry, the language reflects his groundbreaking work. Successive versions, including the Haskell Report 1.1 and version 1.2, introduced nuanced enhancements, shaping Haskell into the language we recognize today.

        \subsection{Haskell in the Digital Age}
            Marking its online presence, Haskell took to the internet in 1994 with haskell.org, a domain that continues to be a hub for enthusiasts. The release of Haskell version 1.3 in 1996 was a landmark, introducing significant refinements and capabilities.

        \subsection{Haskell 98 - A Commitment to Stability}
            The publication of "The Haskell 98 Report: Language and Libraries" in 1999 was a declaration of Haskell's maturity and stability, broadening its appeal and application. This version, later available freely online, cemented Haskell's place in the programming world.

        \subsection{The 21st Century and Haskell Prime}
            With the turn of the century, Haskell faced the need for evolution. The emergence of Haskell Prime (Haskell$'$) signified a modernized, incrementally developed version, with Haskell 2010 being its first notable iteration.

        \subsection{Haskell Today}
            Haskell's narrative continues to unfold, with the Haskell Foundation, established in 2020, championing its development. The support from numerous companies ensures Haskell's enduring presence and evolution in the realm of functional programming.

    \pagebreak
	\section{Features of the Language}
        Haskell is distinguished by a range of unique and powerful features, each contributing to its robustness as a functional programming language.

        \subsection{Purely Functional Programming}
            Haskell is a purely functional language, ensuring functions are free from side effects. This approach simplifies both debugging and testing, as the same input always yields the same output.

        \subsection{Strong, Static Type System with Type Inference}
            The language boasts a robust and static type system, with type checking at compile time to minimize runtime errors. Type inference in Haskell allows for less verbose code while maintaining type safety.

        \subsection{Lazy Evaluation}
            Haskell employs lazy evaluation, delaying computations until their results are needed. This leads to efficient memory utilization and the ability to define infinite data structures.

        \subsection{Immutability}
            Variables in Haskell are immutable. Once a value is assigned to a variable, it cannot be altered, which aids in reducing side effects and simplifying code maintenance.

        \subsection{First-Class Functions}
            Functions in Haskell are treated as first-class citizens, capable of being passed as arguments, returned from other functions, and assigned to variables.

        \subsection{Pattern Matching}
            Haskell offers advanced pattern matching capabilities, allowing for concise and readable code through the direct decomposition of data structures.

        \subsection{High-Level Abstractions}
            The language supports high-level abstractions such as monads, facilitating the management of complex computational patterns like side effects.

        \subsection{Concurrent and Parallel Programming}
            Haskell excels in concurrent and parallel programming, streamlining the development of efficient multi-core processor applications.

        \subsection{Extensive Standard Library}
            Haskell is equipped with a comprehensive standard library, covering a broad spectrum of functionalities, from basic data manipulation to sophisticated algorithms.

        \pagebreak
	\section{Structures of the Language}
	\section{Comparison with Other Languages}
	\section{Simple Code Examples}
	\section{Conclusion}
	\section{References}
	\section{Additional Resources}
	\section{Acknowledgments}
\end{document}
