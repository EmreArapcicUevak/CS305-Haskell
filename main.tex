\documentclass[a4paper, 10pt]{article}

\usepackage[margin = 1in]{geometry} % for spacing around
\usepackage{graphicx} % for including images in your pdfs
\usepackage{xcolor} % for including colors in your pdf
\usepackage{soul} % for text decoration
\usepackage[utf8]{inputenc} % for encoded text
\usepackage[T1]{fontenc}
\usepackage{setspace} % for setting different line spacings between paragrafs.
\usepackage{enumerate} % for letting us get more detailed enumerate lists
\usepackage{multirow} % to let us combine more rows together
\usepackage{colortbl} % for decorating tables
\usepackage{amsmath} % used for representing more complicated math displays
\usepackage{supertabular}
\usepackage{longtable} % both of these packages are used to making really big tables
\usepackage{wrapfig} % allows us to wrap text around figures
\usepackage{fancyhdr} % for making fancy headers
%\usepackage{bibtex} % for making better bibliographies
\usepackage[pdftex]{hyperref} % for letting us make links
\usepackage{lscape} % Allows us to flip from portrait to landspace
\usepackage{tikz} % for high detailed drawing
\usepackage{multicol} % To put things side by side
\usepackage{rotating} % For rotating objects
% \usepackage{draftwatermark} % For adding watermarks
\usepackage{MnSymbol} % for using multiple symbols
\usepackage{mathtools} % Used for more math symbols
\usepackage{xfrac} % For more complciated fractions and to add derivitives
\usepackage{enumitem} % for better enum lists
\usepackage{tcolorbox} % for adding colored text boxes
\usepackage{bm} % Adding bold text to math inputs
\usepackage{pgfplots} % Used for plotting functions
\usepackage{wallpaper} % For adding wallpapers

% Setting up the default image path
\graphicspath{{./Images/}}

% Implementing authro details
\title{Introduction to Haskell Programming Language}
\author{Emre Arapcic-Uevak}
\date{}

% Setting up the fancy page style
\fancypagestyle{customStyle}{
	\lhead{} \chead{} \rhead{}
	\lfoot{} \cfoot{\thepage} \rfoot{}
	\renewcommand{\headrulewidth}{0pt}
	\renewcommand{\footrulewidth}{1pt}
}
\pagestyle{customStyle}

% Setting up hyperref options
\hypersetup {
	colorlinks = false,
	citecolor = black,
	filecolor = blue,
	linkcolor = blue,
	urlcolor = blue,
	pdftex
}

% Custom commands


\begin{document}
	\maketitle
	\vspace{5mm}
	
	\begin{abstract}
		\begin{center}
			This presentation introduces the Haskell programming language, covering its history, key features, data structures, comparisons with other languages, and providing simple code examples. The audience will gain insights into the fundamentals of Haskell and its applications in modern programming.
		\end{center}
	\end{abstract}

	\ThisCenterWallPaper{.4}{Images/IUS Logo.png}

	\pagebreak
	\tableofcontents
	\pagebreak

	\section{Introduction to Haskell}
        \noindent Haskell is a purely functional programming language, distinguished by its strong static typing, high level of abstraction, and lazy evaluation. Inspired by the principles of lambda calculus, Haskell emphasizes functions without side effects and immutable data. Named after the logician Haskell Curry, it stands apart from imperative programming languages by treating computation as the evaluation of mathematical functions, thereby avoiding changing-state and mutable data.
	\section{History and Evolution of Haskell}
	\section{Features of the Language}
	\section{Structures of the Language}
	\section{Comparison with Other Languages}
	\section{Simple Code Examples}
	\section{Conclusion}
	\section{References}
	\section{Additional Resources}
	\section{Acknowledgments}
\end{document}
